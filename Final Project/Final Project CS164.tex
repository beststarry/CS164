\documentclass{article}
\usepackage[left=2cm, right=2cm, top=0cm]{geometry}
\usepackage{amsmath}
\usepackage{amssymb}
\usepackage{fancyvrb}
\usepackage{xcolor}

\usepackage{graphicx}
\usepackage{enumitem}
\usepackage{tikz}
\usepackage{pgfplots}
\usepackage{hyperref}
\setlength\parindent{0pt}
% \hypersetup{
%    colorlinks,
%    citecolor=green,
%    filecolor=black,
%    linkcolor=blue,
%    urlcolor=blue
%}

\begin{document}
\title{Final Project}
\author{Jacob Puthipiroj}
%\date{}
\maketitle

\begin{abstract}
	DTs (Decision trees) are a popular class of predictive modeling, widely used for its interpretability and good performance. However, DTs are not without their drawbacks. DTs can be thought of as a mixed integer programming problem, which allows for combinatorial decisions at each node. Good accuracy can be achieved with MIP decision trees. 
\end{abstract}


% Current algorithms for decision trees, such as CART rely on sequential heuristics



\end{document}